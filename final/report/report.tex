%%%%%%%%%%%%%%%%%%%%%%%%%%%%% Define Article %%%%%%%%%%%%%%%%%%%%%%%%%%%%%%%%%%
\documentclass{ctexart}
%%%%%%%%%%%%%%%%%%%%%%%%%%%%%%%%%%%%%%%%%%%%%%%%%%%%%%%%%%%%%%%%%%%%%%%%%%%%%%%

%%%%%%%%%%%%%%%%%%%%%%%%%%%%% Using Packages %%%%%%%%%%%%%%%%%%%%%%%%%%%%%%%%%%
\usepackage{geometry}
\usepackage{graphicx}
\usepackage{amssymb}
\usepackage{amsmath}
\usepackage{amsthm}
\usepackage{empheq}
\usepackage{mdframed}
\usepackage{booktabs}
\usepackage{lipsum}
\usepackage{graphicx}
\usepackage{color}
\usepackage{psfrag}
\usepackage{pgfplots}
\usepackage{bm}
\usepackage{float}
%%%%%%%%%%%%%%%%%%%%%%%%%%%%%%%%%%%%%%%%%%%%%%%%%%%%%%%%%%%%%%%%%%%%%%%%%%%%%%%

% Other Settings

%%%%%%%%%%%%%%%%%%%%%%%%%% Page Setting %%%%%%%%%%%%%%%%%%%%%%%%%%%%%%%%%%%%%%%
\geometry{a4paper}

%%%%%%%%%%%%%%%%%%%%%%%%%% Define some useful colors %%%%%%%%%%%%%%%%%%%%%%%%%%
\definecolor{ocre}{RGB}{243,102,25}
\definecolor{mygray}{RGB}{243,243,244}
\definecolor{deepGreen}{RGB}{26,111,0}
\definecolor{shallowGreen}{RGB}{235,255,255}
\definecolor{deepBlue}{RGB}{61,124,222}
\definecolor{shallowBlue}{RGB}{235,249,255}
%%%%%%%%%%%%%%%%%%%%%%%%%%%%%%%%%%%%%%%%%%%%%%%%%%%%%%%%%%%%%%%%%%%%%%%%%%%%%%%

%%%%%%%%%%%%%%%%%%%%%%%%%% Define an orangebox command %%%%%%%%%%%%%%%%%%%%%%%%
\newcommand\orangebox[1]{\fcolorbox{ocre}{mygray}{\hspace{1em}#1\hspace{1em}}}
%%%%%%%%%%%%%%%%%%%%%%%%%%%%%%%%%%%%%%%%%%%%%%%%%%%%%%%%%%%%%%%%%%%%%%%%%%%%%%%

%%%%%%%%%%%%%%%%%%%%%%%%%%%% English Environments %%%%%%%%%%%%%%%%%%%%%%%%%%%%%
\newtheoremstyle{mytheoremstyle}{3pt}{3pt}{\normalfont}{0cm}{\rmfamily\bfseries}{}{1em}{{\color{black}\thmname{#1}~\thmnumber{#2}}\thmnote{\,--\,#3}}
\newtheoremstyle{myproblemstyle}{3pt}{3pt}{\normalfont}{0cm}{\rmfamily\bfseries}{}{1em}{{\color{black}\thmname{#1}~\thmnumber{#2}}\thmnote{\,--\,#3}}
\theoremstyle{mytheoremstyle}
\newmdtheoremenv[linewidth=1pt,backgroundcolor=shallowGreen,linecolor=deepGreen,leftmargin=0pt,innerleftmargin=20pt,innerrightmargin=20pt,]{theorem}{Theorem}[section]
\theoremstyle{mytheoremstyle}
\newmdtheoremenv[linewidth=1pt,backgroundcolor=shallowBlue,linecolor=deepBlue,leftmargin=0pt,innerleftmargin=20pt,innerrightmargin=20pt,]{definition}{Definition}[section]
\theoremstyle{myproblemstyle}
\newmdtheoremenv[linecolor=black,leftmargin=0pt,innerleftmargin=10pt,innerrightmargin=10pt,]{problem}{Problem}[section]
%%%%%%%%%%%%%%%%%%%%%%%%%%%%%%%%%%%%%%%%%%%%%%%%%%%%%%%%%%%%%%%%%%%%%%%%%%%%%%%

%%%%%%%%%%%%%%%%%%%%%%%%%%%%%%% Plotting Settings %%%%%%%%%%%%%%%%%%%%%%%%%%%%%
\usepgfplotslibrary{colorbrewer}
\pgfplotsset{width=8cm,compat=1.9}
%%%%%%%%%%%%%%%%%%%%%%%%%%%%%%%%%%%%%%%%%%%%%%%%%%%%%%%%%%%%%%%%%%%%%%%%%%%%%%%

%%%%%%%%%%%%%%%%%%%%%%%%%%%%%%% Title & Author %%%%%%%%%%%%%%%%%%%%%%%%%%%%%%%%
\title{网络数据的概率与统计方法报告}
\author{夏添}
\date{}
%%%%%%%%%%%%%%%%%%%%%%%%%%%%%%%%%%%%%%%%%%%%%%%%%%%%%%%%%%%%%%%%%%%%%%%%%%%%%%%

\begin{document}
    \maketitle
    \newpage

    \section{项目介绍}
本项目旨在对一个真实的社交网络数据集(Twitter Mention Network)进行统计分析。本项目选取了在希格斯玻色子相关的Twitter消息中形成的提及网络数据集。该数据集描述了用户相互提及的关系,从而构成一个复杂的有向图。

本项目的分析主要包括网络的基本统计特性、关键节点的识别、社区结构的发现、网络整体的拓扑性质以及随机模型的模拟。通过这些分析,本项目旨在揭示该社交网络中的信息传播模式、影响力分布和用户群体的组织方式。

\section{网络分析结果}

\subsection{图的基本统计信息}
为了理解网络的基本性质,本项目首先计算了图的一系列基本的统计指标。

\begin{itemize}
    \item \textbf{节点数量: 116408}: 网络中包含了超过11万个用户(节点),表明该话题的讨论规模非常大。
    \item \textbf{边数量: 150818}: 用户之间有15万次提及,构成了网络的基本连接关系。
    \item \textbf{平均度: 2.59}: 每个用户平均与其他约2.6个用户直接相连。这个值相对较低,说明网络可能不是一个高度密集连接的整体。
    \item \textbf{图不连通}: 整个网络并非一个单一的连通图,而是由多个独立的子图构成。这意味着并非所有用户都能通过网络路径相互到达。
    \item \textbf{连通分量: 110704}: 网络中存在大量的独立子图。绝大多数连通分量都非常小,可能只包含少数几个用户的对话。
    \item \textbf{最大连通分量: 1801 节点}: 网络中最大的一个连通子图包含了1801个用户。它代表了网络中最核心的交流群体。
\end{itemize}

这些基本统计数据表明了这一社交网络的特征:规模庞大但结构松散,核心讨论由一个相对较小的最大连通分量主导,而外围则散布着大量孤立的讨论组(连通分量)。

\subsection{中心性分析}
中心性分析旨在识别网络中最重要的节点。本项目从四个不同的维度进行了度量:度中心性、介数中心性、紧密中心性和特征向量中心性。

\subsubsection{总度中心性}
总度中心性计算一个节点的总连接数。
\begin{table}[H]
    \centering
    \caption{总度中心性最高的10个节点}
    \begin{tabular}{|c|c|}
        \hline
        节点 & 总度 \\
        \hline
        88 & 11960 \\
        677 & 3920 \\
        2417 & 2538 \\
        59195 & 1604 \\
        3998 & 1592 \\
        7533 & 1530 \\
        383 & 1358 \\
        1988 & 1191 \\
        13813 & 1067 \\
        519 & 805 \\
        \hline
    \end{tabular}
\end{table}

节点88的度中心性远超其他节点,表明它可能是该话题的核心信息源或交汇点,例如权威新闻机构。其他高中心性节点也是类似的关键角色,但影响力相对较小。

\subsubsection{介数中心性}
介数中心性是网络中所有最短路径经过该节点的次数。
\begin{table}[H]
    \centering
    \caption{介数中心性最高的10个节点 (归一化估计值)}
    \begin{tabular}{|c|c|}
        \hline
        节点 & 介数中心性 (归一化) \\
        \hline
        88 & 1.0000 \\
        64911 & 0.9629 \\
        13808 & 0.8510 \\
        89805 & 0.5423 \\
        12751 & 0.5145 \\
        110903 & 0.4473 \\
        6940 & 0.4363 \\
        67382 & 0.4105 \\
        3998 & 0.3815 \\
        35376 & 0.3550 \\
        \hline
    \end{tabular}
\end{table}
介数中心性高的节点是信息在不同社群之间传播的关键枢纽,高介数中心性的节点在连接不同用户群体方面起重要作用。节点88同样具有最高的介数中心性,进一步证实了其核心地位。

\subsubsection{紧密中心性}
紧密中心性衡量一个节点到网络中所有其他节点的平均距离的倒数。由于原图不连通,本项目在最大的连通分量上计算该指标。
\begin{table}[H]
    \centering
    \caption{紧密中心性最高的10个节点 (在最大连通分量上计算)}
    \begin{tabular}{|c|c|}
        \hline
        节点 & 紧密中心性 \\
        \hline
        88 & 0.4808 \\
        3998 & 0.4060 \\
        89805 & 0.3993 \\
        13808 & 0.3874 \\
        2417 & 0.3785 \\
        5226 & 0.3743 \\
        1276 & 0.3742 \\
        12751 & 0.3736 \\
        26158 & 0.3684 \\
        6241 & 0.3671 \\
        \hline
    \end{tabular}
\end{table}
在最大连通分量中,高紧密中心性的节点能够最快地将信息传递给网络中的其他成员。

\subsubsection{特征向量中心性}

\begin{table}[H]
    \centering
    \caption{特征向量中心性最高的10个节点}
    \begin{tabular}{|c|c|}
        \hline
        节点 & 特征向量中心性 \\
        \hline
        88 & 1.0000 \\
        3998 & 0.5430 \\
        13808 & 0.5284 \\
        13813 & 0.5218 \\
        12751 & 0.2707 \\
        67382 & 0.2605 \\
        5226 & 0.2366 \\
        677 & 0.2214 \\
        4259 & 0.2179 \\
        64911 & 0.2177 \\
        \hline
    \end{tabular}
\end{table}
特征向量中心性再次验证了节点88的高影响力。

\subsection{社区检测}
社区检测旨在发现网络中连接紧密的节点群组,这些群组内部的连接远多于其与网络其余部分的连接。使用Leiden算法。
\begin{itemize}
    \item \textbf{检测到的社区数量: 10681}: 网络中存在大量的小社群,这与连通分量数量相符。
    \item \textbf{分区的模块度: 0.8525}: 模块度值非常高,说明该网络具有明显的社区结构。
\end{itemize}
以下是网络中最大的5个社区及其规模:
\begin{table}[H]
    \centering
    \caption{最大的5个社区及其节点数量}
    \begin{tabular}{|c|c|}
        \hline
        社区ID & 节点数量 \\
        \hline
        0 & 10637 \\
        1 & 6997 \\
        2 & 4897 \\
        3 & 4881 \\
        4 & 4608 \\
        \hline
    \end{tabular}
\end{table}

\subsection{度分布}
度分布描述了网络中节点度的分布情况。对于大型真实网络,度分布通常服从幂律分布。
\begin{figure}[H]
    \centering
    \includegraphics[width=0.8\textwidth]{plots/degree_distribution.png}
    \caption{网络度分布图 (对数-对数坐标)}
    \label{fig:degree_distribution}
\end{figure}
在双对数坐标系下,度分布呈现出近似线性的趋势,这是幂律分布的典型特征,表明该网络是一个无标度网络。

\section{与随机模型的比较}
本项目将推特提及网络与生成的Barabási-Albert (BA)随机模型进行了比较。BA模型是解释无标度网络形成的一种经典概率模型。

\subsection{模型参数与结果对比}
生成了一个与真实网络节点数相同($N=116408$),但受到生成方式影响,平均度整数化的模型。下表对比了两个网络的核心指标。

\begin{table}[H]
    \centering
    \caption{真实网络与BA模型关键指标对比}
    \begin{tabular}{|l|c|c|}
        \hline
        指标 & Higgs 真实网络 & BA 模型 \\
        \hline
        平均度 & 2.59 & 4.00 \\
        \hline
        最大连通分量大小 & 1801 & 1 \\
        \hline
        社区检测模块度 & 0.8503 & 0.5427 \\
        \hline
    \end{tabular}
\end{table}

绘制两个网络的度分布图(见图\ref{fig:degree_distribution_real}和图\ref{fig:degree_distribution_ba})。

\begin{figure}[H]
    \centering
    \begin{minipage}{0.48\textwidth}
        \centering
        \includegraphics[width=\textwidth]{plots/degree_distribution_higgs-mention_network.edgelist.png}
        \caption{真实网络的度分布}
        \label{fig:degree_distribution_real}
    \end{minipage}\hfill
    \begin{minipage}{0.48\textwidth}
        \centering
        \includegraphics[width=\textwidth]{plots/degree_distribution_ba_model.edgelist.png}
        \caption{BA模型的度分布}
        \label{fig:degree_distribution_ba}
    \end{minipage}
\end{figure}

\subsection{对比结论}

\begin{enumerate}
    \item BA模型成功地复现了真实网络无标度的度分布特征。两个网络的度分布图都清晰地呈现了幂律分布的趋势。证明了优先连接是此类社交网络演化的一个基本机制。

    \item 基础的BA模型在模拟此网络时也有明显不足。
    \begin{itemize}
        \item \textbf{未能形成核心社群:} 真实网络拥有一个包含1801个节点的核心连通分量,而BA模型生成的网络则是碎片化的。
        \item \textbf{未能产生紧密的社区结构:} 真实网络的模块度高达0.85,说明其具有非常强大和清晰的社区结构。相比之下,BA模型的模块度仅为0.54。
    \end{itemize}
\end{enumerate}

\section{总结}
通过本次分析,本项目揭示了推特提及网络是一个典型的无标度社交网络。网络规模庞大但结构松散,存在一个核心的巨型连通分量和大量外围小社群。

中心性分析识别出了网络中的关键节点,如节点88,它们在信息传播和网络结构中扮演着核心角色。社区检测则表明网络具有高度模块化的结构。

通过与Barabási-Albert随机模型的比较,验证了优先连接机制形成无标度网络的性质,也表明了基础BA模型在模拟真实网络模型方面的局限性。

\newpage
\end{document}