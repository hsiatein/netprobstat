%%%%%%%%%%%%%%%%%%%%%%%%%%%%% Define Article %%%%%%%%%%%%%%%%%%%%%%%%%%%%%%%%%%
\documentclass{ctexart}
%%%%%%%%%%%%%%%%%%%%%%%%%%%%%%%%%%%%%%%%%%%%%%%%%%%%%%%%%%%%%%%%%%%%%%%%%%%%%%%

%%%%%%%%%%%%%%%%%%%%%%%%%%%%% Using Packages %%%%%%%%%%%%%%%%%%%%%%%%%%%%%%%%%%
\usepackage{geometry}
\usepackage{graphicx}
\usepackage{amssymb}
\usepackage{amsmath}
\usepackage{amsthm}
\usepackage{empheq}
\usepackage{mdframed}
\usepackage{booktabs}
\usepackage{lipsum}
\usepackage{graphicx}
\usepackage{color}
\usepackage{psfrag}
\usepackage{pgfplots}
\usepackage{bm}
%%%%%%%%%%%%%%%%%%%%%%%%%%%%%%%%%%%%%%%%%%%%%%%%%%%%%%%%%%%%%%%%%%%%%%%%%%%%%%%

% Other Settings

%%%%%%%%%%%%%%%%%%%%%%%%%% Page Setting %%%%%%%%%%%%%%%%%%%%%%%%%%%%%%%%%%%%%%%
\geometry{a4paper}

%%%%%%%%%%%%%%%%%%%%%%%%%% Define some useful colors %%%%%%%%%%%%%%%%%%%%%%%%%%
\definecolor{ocre}{RGB}{243,102,25}
\definecolor{mygray}{RGB}{243,243,244}
\definecolor{deepGreen}{RGB}{26,111,0}
\definecolor{shallowGreen}{RGB}{235,255,255}
\definecolor{deepBlue}{RGB}{61,124,222}
\definecolor{shallowBlue}{RGB}{235,249,255}
%%%%%%%%%%%%%%%%%%%%%%%%%%%%%%%%%%%%%%%%%%%%%%%%%%%%%%%%%%%%%%%%%%%%%%%%%%%%%%%

%%%%%%%%%%%%%%%%%%%%%%%%%% Define an orangebox command %%%%%%%%%%%%%%%%%%%%%%%%
\newcommand\orangebox[1]{\fcolorbox{ocre}{mygray}{\hspace{1em}#1\hspace{1em}}}
%%%%%%%%%%%%%%%%%%%%%%%%%%%%%%%%%%%%%%%%%%%%%%%%%%%%%%%%%%%%%%%%%%%%%%%%%%%%%%%

%%%%%%%%%%%%%%%%%%%%%%%%%%%% English Environments %%%%%%%%%%%%%%%%%%%%%%%%%%%%%
\newtheoremstyle{mytheoremstyle}{3pt}{3pt}{\normalfont}{0cm}{\rmfamily\bfseries}{}{1em}{{\color{black}\thmname{#1}~\thmnumber{#2}}\thmnote{\,--\,#3}}
\newtheoremstyle{myproblemstyle}{3pt}{3pt}{\normalfont}{0cm}{\rmfamily\bfseries}{}{1em}{{\color{black}\thmname{#1}~\thmnumber{#2}}\thmnote{\,--\,#3}}
\theoremstyle{mytheoremstyle}
\newmdtheoremenv[linewidth=1pt,backgroundcolor=shallowGreen,linecolor=deepGreen,leftmargin=0pt,innerleftmargin=20pt,innerrightmargin=20pt,]{theorem}{Theorem}[section]
\theoremstyle{mytheoremstyle}
\newmdtheoremenv[linewidth=1pt,backgroundcolor=shallowBlue,linecolor=deepBlue,leftmargin=0pt,innerleftmargin=20pt,innerrightmargin=20pt,]{definition}{Definition}[section]
\theoremstyle{myproblemstyle}
\newmdtheoremenv[linecolor=black,leftmargin=0pt,innerleftmargin=10pt,innerrightmargin=10pt,]{problem}{Problem}[section]
%%%%%%%%%%%%%%%%%%%%%%%%%%%%%%%%%%%%%%%%%%%%%%%%%%%%%%%%%%%%%%%%%%%%%%%%%%%%%%%

%%%%%%%%%%%%%%%%%%%%%%%%%%%%%%% Plotting Settings %%%%%%%%%%%%%%%%%%%%%%%%%%%%%
\usepgfplotslibrary{colorbrewer}
\pgfplotsset{width=8cm,compat=1.9}
%%%%%%%%%%%%%%%%%%%%%%%%%%%%%%%%%%%%%%%%%%%%%%%%%%%%%%%%%%%%%%%%%%%%%%%%%%%%%%%

%%%%%%%%%%%%%%%%%%%%%%%%%%%%%%% Title & Author %%%%%%%%%%%%%%%%%%%%%%%%%%%%%%%%
\title{网络数据的概率与统计方法报告}
\author{夏添}
\date{}
%%%%%%%%%%%%%%%%%%%%%%%%%%%%%%%%%%%%%%%%%%%%%%%%%%%%%%%%%%%%%%%%%%%%%%%%%%%%%%%

\begin{document}
    \maketitle
    \newpage

    \section{网络分析结果}
    \subsection{图的基本统计信息}
    \begin{itemize}
        \item 节点数量: 116408
        \item 边数量: 150818
        \item 平均度: 2.59
        \item 图是否连通: 否
        \item 连通分量的数量: 110704
        \item 最大连通分量的大小: 1801 节点
    \end{itemize}

    \subsection{中心性分析}
    我们计算了网络的总度中心性 (Total Degree Centrality)、介数中心性 (Betweenness Centrality)、紧密中心性 (Closeness Centrality) 和特征向量中心性 (Eigenvector Centrality)。

    \subsubsection{总度中心性}
    总度中心性衡量了节点在网络中的活跃程度。以下是总度中心性最高的10个节点:
    \begin{tabular}{|c|c|}
        \hline
        节点 & 总度中心性 \\
        \hline
        88 & 11960 \\
        677 & 3920 \\
        2417 & 2538 \\
        59195 & 1604 \\
        3998 & 1592 \\
        7533 & 1530 \\
        383 & 1358 \\
        1988 & 1191 \\
        13813 & 1067 \\
        519 & 805 \\
        \hline
    \end{tabular}

    \subsubsection{介数中心性}
    介数中心性衡量了节点在网络中作为“桥梁”的重要性。由于计算量巨大,我们使用了估计值,并进行了归一化处理。以下是介数中心性最高的10个节点:
    \begin{tabular}{|c|c|}
        \hline
        节点 & 介数中心性 (归一化) \\
        \hline
        88 & 1.0000 \\
        64911 & 0.9629 \\
        13808 & 0.8510 \\
        89805 & 0.5423 \\
        12751 & 0.5145 \\
        110903 & 0.4473 \\
        6940 & 0.4363 \\
        67382 & 0.4105 \\
        3998 & 0.3815 \\
        35376 & 0.3550 \\
        \hline
    \end{tabular}

    \subsubsection{紧密中心性}
    紧密中心性衡量了节点到网络中其他所有节点的平均距离。由于原图不连通,我们在最大的连通分量上计算了紧密中心性。以下是紧密中心性最高的10个节点:
    \begin{tabular}{|c|c|}
        \hline
        节点 & 紧密中心性 \\
        \hline
        88 & 0.4808 \\
        3998 & 0.4060 \\
        89805 & 0.3993 \\
        13808 & 0.3874 \\
        2417 & 0.3785 \\
        5226 & 0.3743 \\
        1276 & 0.3742 \\
        12751 & 0.3736 \\
        26158 & 0.3684 \\
        6241 & 0.3671 \\
        \hline
    \end{tabular}

    \subsubsection{特征向量中心性}
    特征向量中心性衡量了节点连接到重要节点的程度。由于图不完全连通,此结果可能需要谨慎解释。以下是特征向量中心性最高的10个节点:
    \begin{tabular}{|c|c|}
        \hline
        节点 & 特征向量中心性 \\
        \hline
        88 & 1.0000 \\
        3998 & 0.5430 \\
        13808 & 0.5284 \\
        13813 & 0.5218 \\
        12751 & 0.2707 \\
        67382 & 0.2605 \\
        5226 & 0.2366 \\
        677 & 0.2214 \\
        4259 & 0.2179 \\
        64911 & 0.2177 \\
        \hline
    \end{tabular}

    \subsection{社区检测}
    我们使用Leiden算法进行了社区检测,以发现网络中的密集连接子结构。
    \begin{itemize}
        \item 检测到的社区数量: 10681
        \item 分区的模块度: 0.8525
    \end{itemize}
    以下是最大的5个社区及其节点数量:
    \begin{tabular}{|c|c|}
        \hline
        社区 & 节点数量 \\
        \hline
        0 & 10637 \\
        1 & 6997 \\
        2 & 4897 \\
        3 & 4881 \\
        4 & 4608 \\
        \hline
    \end{tabular}

    \subsection{度分布}
    度分布图展示了网络中具有不同度数的节点数量。
    \begin{figure}[htbp]
        \centering
        \includegraphics[width=0.8\textwidth]{plots/degree_distribution.png}
        \caption{网络度分布图 (对数-对数坐标)}
        \label{fig:degree_distribution}
    \end{figure}
    \newpage
\end{document}