%%%%%%%%%%%%%%%%%%%%%%%%%%%%% Define Article %%%%%%%%%%%%%%%%%%%%%%%%%%%%%%%%%%
\documentclass{ctexart}
%%%%%%%%%%%%%%%%%%%%%%%%%%%%%%%%%%%%%%%%%%%%%%%%%%%%%%%%%%%%%%%%%%%%%%%%%%%%%%%

%%%%%%%%%%%%%%%%%%%%%%%%%%%%% Using Packages %%%%%%%%%%%%%%%%%%%%%%%%%%%%%%%%%%
\usepackage{geometry}
\usepackage{graphicx}
\usepackage{amssymb}
\usepackage{amsmath}
\usepackage{amsthm}
\usepackage{empheq}
\usepackage{mdframed}
\usepackage{booktabs}
\usepackage{lipsum}
\usepackage{graphicx}
\usepackage{color}
\usepackage{psfrag}
\usepackage{pgfplots}
\usepackage{bm}
%%%%%%%%%%%%%%%%%%%%%%%%%%%%%%%%%%%%%%%%%%%%%%%%%%%%%%%%%%%%%%%%%%%%%%%%%%%%%%%

% Other Settings

%%%%%%%%%%%%%%%%%%%%%%%%%% Page Setting %%%%%%%%%%%%%%%%%%%%%%%%%%%%%%%%%%%%%%%
\geometry{a4paper}

%%%%%%%%%%%%%%%%%%%%%%%%%% Define some useful colors %%%%%%%%%%%%%%%%%%%%%%%%%%
\definecolor{ocre}{RGB}{243,102,25}
\definecolor{mygray}{RGB}{243,243,244}
\definecolor{deepGreen}{RGB}{26,111,0}
\definecolor{shallowGreen}{RGB}{235,255,255}
\definecolor{deepBlue}{RGB}{61,124,222}
\definecolor{shallowBlue}{RGB}{235,249,255}
%%%%%%%%%%%%%%%%%%%%%%%%%%%%%%%%%%%%%%%%%%%%%%%%%%%%%%%%%%%%%%%%%%%%%%%%%%%%%%%

%%%%%%%%%%%%%%%%%%%%%%%%%% Define an orangebox command %%%%%%%%%%%%%%%%%%%%%%%%
\newcommand\orangebox[1]{\fcolorbox{ocre}{mygray}{\hspace{1em}#1\hspace{1em}}}
%%%%%%%%%%%%%%%%%%%%%%%%%%%%%%%%%%%%%%%%%%%%%%%%%%%%%%%%%%%%%%%%%%%%%%%%%%%%%%%

%%%%%%%%%%%%%%%%%%%%%%%%%%%% English Environments %%%%%%%%%%%%%%%%%%%%%%%%%%%%%
\newtheoremstyle{mytheoremstyle}{3pt}{3pt}{\normalfont}{0cm}{\rmfamily\bfseries}{}{1em}{{\color{black}\thmname{#1}~\thmnumber{#2}}\thmnote{\,--\,#3}}
\newtheoremstyle{myproblemstyle}{3pt}{3pt}{\normalfont}{0cm}{\rmfamily\bfseries}{}{1em}{{\color{black}\thmname{#1}~\thmnumber{#2}}\thmnote{\,--\,#3}}
\theoremstyle{mytheoremstyle}
\newmdtheoremenv[linewidth=1pt,backgroundcolor=shallowGreen,linecolor=deepGreen,leftmargin=0pt,innerleftmargin=20pt,innerrightmargin=20pt,]{theorem}{Theorem}[section]
\theoremstyle{mytheoremstyle}
\newmdtheoremenv[linewidth=1pt,backgroundcolor=shallowBlue,linecolor=deepBlue,leftmargin=0pt,innerleftmargin=20pt,innerrightmargin=20pt,]{definition}{Definition}[section]
\theoremstyle{myproblemstyle}
\newmdtheoremenv[linecolor=black,leftmargin=0pt,innerleftmargin=10pt,innerrightmargin=10pt,]{problem}{Problem}[section]
%%%%%%%%%%%%%%%%%%%%%%%%%%%%%%%%%%%%%%%%%%%%%%%%%%%%%%%%%%%%%%%%%%%%%%%%%%%%%%%

%%%%%%%%%%%%%%%%%%%%%%%%%%%%%%% Plotting Settings %%%%%%%%%%%%%%%%%%%%%%%%%%%%%
\usepgfplotslibrary{colorbrewer}
\pgfplotsset{width=8cm,compat=1.9}
%%%%%%%%%%%%%%%%%%%%%%%%%%%%%%%%%%%%%%%%%%%%%%%%%%%%%%%%%%%%%%%%%%%%%%%%%%%%%%%

%%%%%%%%%%%%%%%%%%%%%%%%%%%%%%% Title & Author %%%%%%%%%%%%%%%%%%%%%%%%%%%%%%%%
\title{网络数据的概率与统计方法报告}
\author{夏添}
\date{}
%%%%%%%%%%%%%%%%%%%%%%%%%%%%%%%%%%%%%%%%%%%%%%%%%%%%%%%%%%%%%%%%%%%%%%%%%%%%%%%

\begin{document}
    \maketitle
    \newpage

    \section{项目介绍}
本项目旨在对一个真实的社交网络数据集进行深入的概率与统计分析。我们选取了在“希格斯玻色子”相关的Twitter消息中形成的提及网络 (mention network) 数据集。该数据集捕捉了用户之间通过“@”符号相互提及的关系,从而构成一个复杂的有向图。

我们的分析将涵盖网络的基本统计特性、关键节点的识别、社区结构的发现以及网络整体的拓扑性质。通过这些分析,我们旨在揭示该社交网络中的信息传播模式、影响力分布和用户群体的组织方式。

\section{网络分析结果}

\subsection{图的基本统计信息}
为了从宏观上理解网络的规模和连通性,我们首先计算了图的一系列基本统计指标。

\begin{itemize}
    \item \textbf{节点数量: 116408}: 网络中包含了超过11万个独立的用户(节点),这表明该话题的讨论规模非常庞大。
    \item \textbf{边数量: 150818}: 用户之间产生了约15万次提及,构成了网络的基本连接。
    \item \textbf{平均度: 2.59}: 每个用户平均与其他约2.6个用户直接相连。这个值相对较低,暗示网络可能不是一个高度密集连接的整体。
    \item \textbf{图是否连通: 否}: 整个网络并非一个单一的连通图,而是由多个独立的子图构成。这意味着并非所有用户都能通过网络路径相互到达。
    \item \textbf{连通分量的数量: 110704}: 网络中存在大量的独立子图。绝大多数连通分量都非常小,可能只包含少数几个用户的孤立对话。
    \item \textbf{最大连通分量的大小: 1801 节点}: 网络中最大的一个连通子图包含了1801个用户。这可以被视为主干讨论区,大部分有意义的分析(如紧密中心性)将在这个子图上进行,因为它代表了网络中最核心的交流群体。
\end{itemize}

这些基本统计数据揭示了一个典型的大型社交网络的特征:规模庞大但结构松散,核心讨论由一个相对较小的“巨型连通分量”主导,而外围则散布着大量孤立的对话。

\subsection{中心性分析}
中心性分析旨在识别网络中最重要的节点。我们从四个不同的维度进行了度量:度中心性、介数中心性、紧密中心性和特征向量中心性。

\subsubsection{总度中心性}
总度中心性计算一个节点的总连接数(入度与出度之和),是衡量节点活跃度的最直接指标。
\begin{table}[htbp]
    \centering
    \caption{总度中心性最高的10个节点}
    \begin{tabular}{|c|c|}
        \hline
        节点 & 总度 \\
        \hline
        88 & 11960 \\
        677 & 3920 \\
        2417 & 2538 \\
        59195 & 1604 \\
        3998 & 1592 \\
        7533 & 1530 \\
        383 & 1358 \\
        1988 & 1191 \\
        13813 & 1067 \\
        519 & 805 \\
        \hline
    \end{tabular}
\end{table}
\textbf{解读}: 节点88的度中心性远超其他节点,表明它可能是该话题的核心信息源或交汇点,例如一个权威新闻机构、研究中心或极具影响力的公众人物。其他高中心性节点也扮演着类似的关键角色,但影响力相对较小。

\subsubsection{介数中心性}
介数中心性衡量一个节点在网络中作为“桥梁”的能力,即网络中所有最短路径经过该节点的次数。
\begin{table}[htbp]
    \centering
    \caption{介数中心性最高的10个节点 (归一化估计值)}
    \begin{tabular}{|c|c|}
        \hline
        节点 & 介数中心性 (归一化) \\
        \hline
        88 & 1.0000 \\
        64911 & 0.9629 \\
        13808 & 0.8510 \\
        89805 & 0.5423 \\
        12751 & 0.5145 \\
        110903 & 0.4473 \\
        6940 & 0.4363 \\
        67382 & 0.4105 \\
        3998 & 0.3815 \\
        35376 & 0.3550 \\
        \hline
    \end{tabular}
\end{table}
\textbf{解读}: 介数中心性高的节点是信息在不同社群之间传播的关键枢纽。节点88同样具有最高的介数中心性,进一步证实了其核心地位。其他高介数中心性的节点(如64911、13808)虽然总度数不一定最高,但它们在连接不同用户群体方面起着至关重要的作用。

\subsubsection{紧密中心性}
紧密中心性衡量一个节点到网络中所有其他节点的平均距离的倒数。值越高,表明该节点越容易“接触”到网络中的其他所有节点。由于原图不连通,我们在最大的连通分量上计算了该指标。
\begin{table}[htbp]
    \centering
    \caption{紧密中心性最高的10个节点 (在最大连通分量上计算)}
    \begin{tabular}{|c|c|}
        \hline
        节点 & 紧密中心性 \\
        \hline
        88 & 0.4808 \\
        3998 & 0.4060 \\
        89805 & 0.3993 \\
        13808 & 0.3874 \\
        2417 & 0.3785 \\
        5226 & 0.3743 \\
        1276 & 0.3742 \\
        12751 & 0.3736 \\
        26158 & 0.3684 \\
        6241 & 0.3671 \\
        \hline
    \end{tabular}
\end{table}
\textbf{解读}: 在核心讨论区(最大连通分量)中,这些高紧密中心性的节点处于信息传播的有利位置,能够最快地将信息传递给网络中的其他大部分成员。

\subsubsection{特征向量中心性}
特征向量中心性不仅考虑节点的连接数量,还考虑其邻居节点的重要性。一个与许多重要节点相连的节点,其特征向量中心性会更高。
\begin{table}[htbp]
    \centering
    \caption{特征向量中心性最高的10个节点}
    \begin{tabular}{|c|c|}
        \hline
        节点 & 特征向量中心性 \\
        \hline
        88 & 1.0000 \\
        3998 & 0.5430 \\
        13808 & 0.5284 \\
        13813 & 0.5218 \\
        12751 & 0.2707 \\
        67382 & 0.2605 \\
        5226 & 0.2366 \\
        677 & 0.2214 \\
        4259 & 0.2179 \\
        64911 & 0.2177 \\
        \hline
    \end{tabular}
\end{table}
\textbf{解读}: 特征向量中心性再次验证了节点88的顶级影响力。同时,它也识别出了一些虽然度数不是最高,但因其连接到其他重要节点而具有高影响力的节点(例如3998、13808)。

\subsection{社区检测}
社区检测旨在发现网络中连接紧密的节点群组,这些群组内部的连接远多于其与网络其余部分的连接。我们使用了高效的Leiden算法。
\begin{itemize}
    \item \textbf{检测到的社区数量: 10681}: 网络中存在大量的小社群,这与连通分量数量巨大相符。
    \item \textbf{分区的模块度: 0.8525}: 模块度值非常高(通常大于0.3即认为有显著的社区结构),表明该网络具有非常清晰和稳固的社区结构。
\end{itemize}
以下是网络中最大的5个社区及其规模:
\begin{table}[htbp]
    \centering
    \caption{最大的5个社区及其节点数量}
    \begin{tabular}{|c|c|}
        \hline
        社区ID & 节点数量 \\
        \hline
        0 & 10637 \\
        1 & 6997 \\
        2 & 4897 \\
        3 & 4881 \\
        4 & 4608 \\
        \hline
    \end{tabular}
\end{table}
\textbf{解读}: 尽管网络中有上万个社区,但存在少数几个规模巨大的社区,这些社区可能代表了围绕特定子话题、语言或用户群体的讨论核心。

\subsection{度分布}
度分布描述了网络中节点度的分布情况。对于大型真实网络,度分布通常服从“幂律分布”,即少数节点拥有极高的度,而大多数节点的度很低。
\begin{figure}[htbp]
    \centering
    \includegraphics[width=0.8\textwidth]{plots/degree_distribution.png}
    \caption{网络度分布图 (对数-对数坐标)}
    \label{fig:degree_distribution}
\end{figure}
\textbf{解读}: 在双对数坐标系下,度分布呈现出近似线性的趋势,这是幂律分布的典型特征,表明该网络是一个“无标度网络”(Scale-Free Network)。这种结构意味着网络对随机的节点移除具有鲁棒性,但对针对高中心性节点的蓄意攻击则非常脆弱。

\section{总结}
通过本次分析,我们揭示了“希格斯玻色子”提及网络是一个典型的无标度社交网络。网络规模庞大但结构松散,存在一个核心的“巨型连通分量”和大量外围小社群。
中心性分析识别出了网络中的关键节点(如节点88),它们在信息传播和网络结构中扮演着核心角色。社区检测则表明网络具有高度模块化的结构。
这些发现为理解在线社交网络中的信息流动和影响力构建提供了有价值的见解。
\newpage
\end{document}