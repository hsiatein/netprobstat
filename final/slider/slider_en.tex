\documentclass{beamer}

% --- Theme and Color ---
\usetheme{Madrid}
\usecolortheme{default}

% --- Packages ---
\usepackage{graphicx} % For including images
\usepackage{booktabs} % For beautiful tables
\usepackage{colortbl} % For table colors

% --- Document Information ---
\title{Final Project}
\subtitle{Analysis of Twitter Mention Network}
\author{Xia Tian}
\date{}

\begin{document}

% --- Title Page ---
\begin{frame}
    \titlepage
\end{frame}

% --- Table of Contents ---
\begin{frame}
    \frametitle{Outline}
    \tableofcontents
\end{frame}

% --- Sections ---
\section{Project Introduction}
\begin{frame}
    \frametitle{Project Introduction}
    \begin{itemize}

        \item \textbf{Dataset:} The Twitter Mention Network, a mention network related to the Higgs boson.
        \begin{itemize}
            \item Records the relationships between users who mention each other.
            \item Forms a complex directed graph.
        \end{itemize}
        \item \textbf{Analysis Content:}
        \begin{itemize}
            \item Basic statistical properties of the network
            \item Centrality analysis
            \item Community structure
            \item Degree distribution
            \item Comparison with stochastic models
        \end{itemize}
    \end{itemize}
\end{frame}

\section{Network Analysis Results}
\subsection{Basic Statistical Information}
\begin{frame}
    \frametitle{Basic Statistical Information of the Graph}
    \begin{itemize}
        \item \textbf{Nodes:} 116,408
        \item \textbf{Edges:} 150,818
        \item \textbf{Average Degree:} 2.59
        \item \textbf{Connected?} No
        \item \textbf{Connected Components:} 110,704
        \item \textbf{Giant Connected Component:} 1,801
    \end{itemize}
    
    \vfill
\end{frame}

\subsection{Centrality Analysis}
\begin{frame}
    \frametitle{Centrality Analysis}
    \begin{columns}[T]
        \begin{column}{0.5\textwidth}
            \textbf{Degree Centrality}
            \begin{tabular}{|c|c|}
                \hline
                Node & Degree \\
                \hline
                88 & 11960 \\
                677 & 3920 \\
                2417 & 2538 \\
                \hline
            \end{tabular}
            \vfill
            \textbf{Closeness Centrality}

            \begin{tabular}{|c|c|}
                \hline
                Node & Closeness \\
                \hline
                88 & 0.4808 \\
                3998 & 0.4060 \\
                89805 & 0.3993 \\
                \hline
            \end{tabular}
        \end{column}
        
        \begin{column}{0.5\textwidth}
            \textbf{Betweenness Centrality}
            \begin{tabular}{|c|c|}
                \hline
                Node & Betweenness \\
                \hline
                88 & 1.00 \\
                64911 & 0.96 \\
                13808 & 0.85 \\
                \hline
            \end{tabular}
            \vfill
            \textbf{Eigenvector Centrality}
            \begin{tabular}{|c|c|}
                \hline
                Node & Eigenvector \\
                \hline
                88 & 1.00 \\
                3998 & 0.54 \\
                13808 & 0.53 \\
                \hline
            \end{tabular}
        \end{column}
    \end{columns}
\end{frame}

\subsection{Community Detection}
\begin{frame}
    \frametitle{Community Detection}
    \begin{itemize}
        \item \textbf{Algorithm:} Leiden Algorithm
        \item \textbf{Communities:} 10,681
        
        \item \textbf{Modularity:} 0.8525
        \begin{itemize}
            \item This is a very high value. Indicates that the network has a significant community structure.
        \end{itemize}
        
        \item \textbf{5 Largest Communities:}
            \begin{center}
            \begin{tabular}{cc}
                \toprule
                Community ID & Number of Nodes \\
                \midrule
                0 & 10637 \\
                1 & 6997 \\
                2 & 4897 \\
                3 & 4881 \\
                4 & 4608 \\
                \bottomrule
            \end{tabular}
            \end{center}
    \end{itemize}
\end{frame}

\subsection{Degree Distribution}
\begin{frame}
    \frametitle{Degree Distribution}
    \begin{figure}
        \includegraphics[width=0.9\textwidth]{../report/plots/degree_distribution.png}
        \caption{Degree Distribution (log-log scale)}
    \end{figure}
\end{frame}

\section{Comparison with Stochastic Model}
\begin{frame}
    \frametitle{Comparison with Stochastic Model}
    \begin{itemize}
        \item \textbf{Model:} Barabási-Albert (BA) model.
        \item \textbf{Parameters:}
        \begin{itemize}
            \item $N = 116,408$
            \item $m=2$
        \end{itemize}
    \end{itemize}
    
    \begin{beamercolorbox}[sep=1em]{block body}
    \begin{tabular}{l c c}
        \toprule
        \textbf{Metric} & \textbf{Higgs Network} & \textbf{BA Model} \\
        \midrule
        Average Degree & 2.59 & 4.00 \\
        GCC Size & 1801 & 1 \\
        Modularity & 0.8525 & 0.5427 \\
        \bottomrule
    \end{tabular}
    \end{beamercolorbox}
\end{frame}

\begin{frame}
    \frametitle{Comparison with Stochastic Model}
    \begin{columns}[T]
        \begin{column}{0.5\textwidth}
            \textbf{Real Network}
            \includegraphics[width=\textwidth]{../report/plots/degree_distribution_higgs-mention_network.edgelist.png}
        \end{column}
        \begin{column}{0.5\textwidth}
            \textbf{BA Model}
            \includegraphics[width=\textwidth]{../report/plots/degree_distribution_ba_model.edgelist.png}
        \end{column}
    \end{columns}
\end{frame}

\begin{frame}
    \frametitle{Comparison Conclusions}
    \begin{itemize}
        \item \textbf{Success of the BA Model:}
        \begin{itemize}
            \item Reproduces the scale-free nature. Both show a power-law degree distribution.
            \item Confirms that preferential attachment is important to the evolution of this social network.
        \end{itemize}
        
        \item \textbf{Limitations of the BA Model:}
        \begin{itemize}
            \item Failed to generate a core community.
            \item Failed to create a strong community structure.
        \end{itemize}
    \end{itemize}
\end{frame}

% --- End Page ---
\begin{frame}
    \centering
    \vfill
    \Huge Thank you!
    \vfill
\end{frame}

\end{document}
