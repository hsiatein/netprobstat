\documentclass{beamer}

% --- Theme and Color ---
\usetheme{Madrid}
\usecolortheme{default}

% --- Packages ---
\usepackage{graphicx} % For including images
\usepackage{booktabs} % For beautiful tables
\usepackage{colortbl} % For table colors

% --- Document Information ---
\title{Probabilistic and Statistical Analysis of Network Data}
\subtitle{A Case Study of the Higgs-Mention Network}
\author{Tian Xia}
\date{}

\begin{document}

% --- Title Page ---
\begin{frame}
    \titlepage
\end{frame}

% --- Table of Contents ---
\begin{frame}
    \frametitle{Outline}
    \tableofcontents
\end{frame}

% --- Sections ---
\section{Project Introduction}
\begin{frame}
    \frametitle{Project Introduction}
    \begin{itemize}
        \item \textbf{Objective:} To conduct an in-depth probabilistic and statistical analysis of a real-world social network dataset.
        \item \textbf{Dataset:} The Higgs-Mention Network, a Twitter mention network related to the Higgs boson.
        \begin{itemize}
            \item Captures the relationships between users who mention each other using the "@" symbol.
            \item Forms a complex directed graph.
        \end{itemize}
        \item \textbf{Analysis Content:}
        \begin{itemize}
            \item Basic statistical properties of the network
            \item Identification of key nodes (centrality analysis)
            \item Discovery of community structure
            \item Topological properties of the network (degree distribution)
        \end{itemize}
    \end{itemize}
\end{frame}

\section{Network Analysis Results}
\subsection{Basic Statistical Information}
\begin{frame}
    \frametitle{Basic Statistical Information of the Graph}
    \begin{itemize}
        \item \textbf{Number of Nodes:} 116,408 (large-scale discussion)
        \item \textbf{Number of Edges:} 150,818
        \item \textbf{Average Degree:} 2.59 (the network is relatively sparse)
        \pause
        \item \textbf{Is the graph connected?} \textcolor{red}{No}
        \item \textbf{Number of Connected Components:} 110,704 (a large number of independent subgraphs)
        \pause
        \item \textbf{Giant Connected Component (GCC):}
        \begin{itemize}
            \item \textbf{Size:} 1,801 nodes
            \item \textbf{Significance:} The core communication group in the network, and the focus of subsequent analysis.
        \end{itemize}
    \end{itemize}
    \vfill
    \small\textit{Conclusion: A typical real-world social network—large in scale, loosely structured, consisting of a core connected component and a large number of isolated conversations.}
\end{frame}

\subsection{Centrality Analysis}
\begin{frame}
    \frametitle{Centrality Analysis (1/2) - Identifying Key Nodes}
    \begin{columns}[T]
        \begin{column}{0.5\textwidth}
            \textbf{Degree Centrality (Activity)}
            \begin{tabular}{|c|c|}
                \hline
                Node & Degree \\
                \hline
                \rowcolor{yellow!30} \textbf{88} & \textbf{11960} \\
                677 & 3920 \\
                2417 & 2538 \\
                \hline
            \end{tabular}
            \pause
            \vfill
            \textbf{Closeness Centrality (Information Spread Efficiency)}
            \tiny{(Calculated on the Giant Connected Component)}
            \begin{tabular}{|c|c|}
                \hline
                Node & Closeness \\
                \hline
                \rowcolor{yellow!30} \textbf{88} & \textbf{0.4808} \\
                3998 & 0.4060 \\
                89805 & 0.3993 \\
                \hline
            \end{tabular}
        \end{column}
        
        \begin{column}{0.5\textwidth}
            \textbf{Betweenness Centrality (Bridge Role)}
            \begin{tabular}{|c|c|}
                \hline
                Node & Betweenness \\
                \hline
                \rowcolor{yellow!30} \textbf{88} & \textbf{1.00} \\
                64911 & 0.96 \\
                13808 & 0.85 \\
                \hline
            \end{tabular}
            \pause
            \vfill
            \textbf{Eigenvector Centrality (Neighbor Importance)}
            \begin{tabular}{|c|c|}
                \hline
                Node & Eigenvector \\
                \hline
                \rowcolor{yellow!30} \textbf{88} & \textbf{1.00} \\
                3998 & 0.54 \\
                13808 & 0.53 \\
                \hline
            \end{tabular}
        \end{column}
    \end{columns}
\end{frame}

\begin{frame}
    \frametitle{Centrality Analysis (2/2) - Key Findings}
    \centering
        \huge\textbf{Node 88}
        \vskip1cm
        \Large
        Ranks \textbf{\textcolor{red}{first}} in all centrality measures
    \pause
    \begin{itemize}
        \item \textbf{Highest Degree:} The most active participant in the network.
        \item \textbf{Highest Betweenness:} The core hub for information dissemination between different communities.
        \item \textbf{Highest Closeness:} In the most advantageous position for information spread in the core discussion area.
        \item \textbf{Highest Eigenvector:} Connects a large number of other important nodes, with huge influence.
    \end{itemize}
    \vfill
    \Large\textit{Inference: Node 88 may be an authoritative news agency, research center, or a highly influential public figure.}
\end{frame}

\subsection{Community Detection}
\begin{frame}
    \frametitle{Community Detection - Discovering Groups in the Network}
    \begin{itemize}
        \item \textbf{Algorithm:} Leiden Algorithm (efficient and commonly used)
        \item \textbf{Number of Communities Detected:} 10,681
        \pause
        \item \textbf{Modularity of the Partition:} \textbf{0.8525}
        \begin{itemize}
            \item This is a very high value (typically > 0.3 is considered significant).
            \item Indicates that the network has a \textbf{very clear and stable} community structure.
        \end{itemize}
        \pause
        \item \textbf{Sizes of the 5 Largest Communities:}
            \begin{center}
            \begin{tabular}{cc}
                \toprule
                Community ID & Number of Nodes \\
                \midrule
                0 & 10637 \\
                1 & 6997 \\
                2 & 4897 \\
                3 & 4881 \\
                4 & 4608 \\
                \bottomrule
            \end{tabular}
            \end{center}
    \end{itemize}
\end{frame}

\subsection{Degree Distribution}
\begin{frame}
    \frametitle{Degree Distribution - Power Law and Scale-Free Networks}
    \begin{figure}
        \includegraphics[width=0.9\textwidth]{../report/plots/degree_distribution.png}
        \caption{Degree Distribution of the Network (log-log scale)}
    \end{figure}
    \pause
    \begin{itemize}
        \item On a log-log scale, the degree distribution shows an approximately \textbf{linear} trend.
        \item This is a typical feature of a \textbf{Power Law} distribution.
        \item \textbf{Conclusion:} The network is a "Scale-Free Network".
        \begin{itemize}
            \item \textbf{Characteristics:} A few nodes have a very high number of connections, while most nodes have a very low number of connections.
            \item \textbf{Properties:} Resilient to random failures, but vulnerable to attacks targeting core nodes (like node 88).
        \end{itemize}
    \end{itemize}
\end{frame}

\section{Summary}
\begin{frame}
    \frametitle{Summary and Findings}
    \begin{itemize}
        \item \textbf{Network Type:}
        The network is a typical \textbf{Scale-Free} social network, large in scale but loosely structured.
        \pause
        \item \textbf{Network Structure:}
        \begin{itemize}
            \item There is a core "Giant Connected Component" (GCC) and a large number of peripheral small communities.
            \item It has a highly modular (community) structure (modularity > 0.85).
        \end{itemize}
        \pause
        \item \textbf{Key Nodes:}
        Centrality analysis successfully identified the core nodes in the network (like \textbf{Node 88}), which play a key role in information dissemination and network structure stability.
        \pause
        \item \textbf{Implications:}
        These findings provide valuable insights into understanding information flow, influence building, and community organization in online social networks.
    \end{itemize}
\end{frame}

% --- End Page ---
\begin{frame}
    \centering
    \vfill
    \Huge Thank you for your attention!
    \vfill
    \Large Q \& A
    \vfill
\end{frame}

\end{document}
