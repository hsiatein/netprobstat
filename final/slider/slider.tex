\documentclass[UTF8]{beamer}

% --- 主题与配色 ---
\usetheme{Madrid}
\usecolortheme{default}

% --- 依赖包 ---
\usepackage{graphicx} % 用于插入图片
\usepackage{booktabs} % 用于美化表格
\usepackage{colortbl} % 用于表格颜色

% --- 文稿信息 ---
\title{网络数据的概率与统计分析}
\subtitle{以Higgs-Mention Network为例}
\author{夏添}
\institute{Math6116P Course Project}
\date{\today}

\begin{document}

% --- 标题页 ---
\begin{frame}
    \titlepage
\end{frame}

% --- 目录页 ---
\begin{frame}
    \frametitle{内容大纲}
    \tableofcontents
\end{frame}

% --- 各个章节 ---
\section{项目介绍}
\begin{frame}
    \frametitle{项目介绍}
    \begin{itemize}
        \item \textbf{目标:} 对一个真实的社交网络数据集进行深入的概率与统计分析。
        \item \textbf{数据集:} “希格斯玻色子”相关的Twitter提及网络 (Higgs-Mention Network)。
        \begin{itemize}
            \item 捕捉用户之间通过“@”符号相互提及的关系。
            \item 构成一个复杂的有向图。
        \end{itemize}
        \item \textbf{分析内容:}
        \begin{itemize}
            \item 网络基本统计特性
            \item 关键节点识别 (中心性分析)
            \item 社区结构发现
            \item 网络拓扑性质 (度分布)
        \end{itemize}
    \end{itemize}
\end{frame}

\section{网络分析结果}
\subsection{基本统计信息}
\begin{frame}
    \frametitle{图的基本统计信息}
    \begin{itemize}
        \item \textbf{节点数量:} 116,408 (讨论规模庞大)
        \item \textbf{边数量:} 150,818
        \item \textbf{平均度:} 2.59 (网络相对稀疏)
        \pause
        \item \textbf{图是否连通:} \textcolor{red}{否}
        \item \textbf{连通分量数量:} 110,704 (存在大量独立子图)
        \pause
        \item \textbf{最大连通分量 (GCC):}
        \begin{itemize}
            \item \textbf{大小:} 1,801 节点
            \item \textbf{意义:} 网络中最核心的交流群体,是后续部分分析的核心。
        \end{itemize}
    \end{itemize}
    \vfill
    \small\textit{结论: 典型的真实社交网络——规模大、结构松散,由一个核心连通分量和大量孤立对话构成。}
\end{frame}

\subsection{中心性分析}
\begin{frame}
    \frametitle{中心性分析 (1/2) - 识别关键节点}
    \begin{columns}[T]
        \begin{column}{0.5\textwidth}
            \textbf{总度中心性 (活跃度)}
            \begin{tabular}{|c|c|}
                \hline
                节点 & 总度 \\
                \hline
                \rowcolor{yellow!30} \textbf{88} & \textbf{11960} \\
                677 & 3920 \\
                2417 & 2538 \\
                \hline
            \end{tabular}
            \pause
            \vfill
            \textbf{紧密中心性 (信息传播效率)}
            \tiny{(在最大连通分量上计算)}
            \begin{tabular}{|c|c|}
                \hline
                节点 & 紧密中心性 \\
                \hline
                \rowcolor{yellow!30} \textbf{88} & \textbf{0.4808} \\
                3998 & 0.4060 \\
                89805 & 0.3993 \\
                \hline
            \end{tabular}
        \end{column}
        
        \begin{column}{0.5\textwidth}
            \textbf{介数中心性 (桥梁作用)}
            \begin{tabular}{|c|c|}
                \hline
                节点 & 介数 \\
                \hline
                \rowcolor{yellow!30} \textbf{88} & \textbf{1.00} \\
                64911 & 0.96 \\
                13808 & 0.85 \\
                \hline
            \end{tabular}
            \pause
            \vfill
            \textbf{特征向量中心性 (邻居重要性)}
            \begin{tabular}{|c|c|}
                \hline
                节点 & 特征向量 \\
                \hline
                \rowcolor{yellow!30} \textbf{88} & \textbf{1.00} \\
                3998 & 0.54 \\
                13808 & 0.53 \\
                \hline
            \end{tabular}
        \end{column}
    \end{columns}
\end{frame}

\begin{frame}
    \frametitle{中心性分析 (2/2) - 关键发现}
    \begin{beamerbody}
        \centering
        \huge\textbf{节点 88}
        \vskip1cm
        \Large
        在所有中心性度量中均\textbf{\textcolor{red}{排名第一}}
    \end{beamerbody}
    \pause
    \begin{itemize}
        \item \textbf{总度最高:} 网络中最活跃的参与者。
        \item \textbf{介数最高:} 信息在不同社群间传播的核心枢纽。
        \item \textbf{紧密中心性最高:} 在核心讨论区中处于信息传播的最有利位置。
        \item \textbf{特征向量中心性最高:} 连接了大量其他重要节点,影响力巨大。
    \end{itemize}
    \vfill
    \Large\textit{推断: 节点 88 可能是权威新闻机构、研究中心或极具影响力的公众人物。}
\end{frame}

\subsection{社区检测}
\begin{frame}
    \frametitle{社区检测 - 发现网络中的群组}
    \begin{itemize}
        \item \textbf{算法:} Leiden 算法 (高效且常用)
        \item \textbf{检测到的社区数量:} 10,681
        \pause
        \item \textbf{分区的模块度:} \textbf{0.8525}
        \begin{itemize}
            \item 这是一个非常高的值 (通常 > 0.3 即认为有意义)。
            \item 表明该网络具有\textbf{非常清晰和稳固}的社区结构。
        \end{itemize}
        \pause
        \item \textbf{最大的5个社区规模:}
            \begin{center}
            \begin{tabular}{cc}
                \toprule
                社区 ID & 节点数量 \\
                \midrule
                0 & 10637 \\
                1 & 6997 \\
                2 & 4897 \\
                3 & 4881 \\
                4 & 4608 \\
                \bottomrule
            \end{tabular}
            \end{center}
    \end{itemize}
\end{frame}

\subsection{度分布}
\begin{frame}
    \frametitle{度分布 - 幂律与无标度网络}
    \begin{figure}
        \includegraphics[width=0.9\textwidth]{../report/plots/degree_distribution.png}
        \caption{网络度分布图 (对数-对数坐标)}
    \end{figure}
    \pause
    \begin{itemize}
        \item 在双对数坐标下,度分布呈现近似\textbf{线性}趋势。
        \item 这是\textbf{幂律分布 (Power Law)} 的典型特征。
        \item \textbf{结论:} 该网络是一个“无标度网络”(Scale-Free Network)。
        \begin{itemize}
            \item \textbf{特点:} 少数节点拥有极高连接数,大多数节点连接数很低。
            \item \textbf{性质:} 对随机故障有弹性,但对针对核心节点(如节点88)的攻击很脆弱。
        \end{itemize}
    \end{itemize}
\end{frame}

\section{总结}
\begin{frame}
    \frametitle{总结与发现}
    \begin{itemize}
        \item \textbf{网络类型:}
        该网络是一个典型的\textbf{无标度 (Scale-Free)}社交网络,规模庞大但结构松散。
        \pause
        \item \textbf{网络结构:}
        \begin{itemize}
            \item 存在一个核心的“巨型连通分量”(GCC) 和大量外围小社群。
            \item 具有高度模块化(社区化)的结构 (模块度 > 0.85)。
        \end{itemize}
        \pause
        \item \textbf{关键节点:}
        中心性分析成功识别出网络中的核心节点 (如\textbf{节点 88}),它们在信息传播和网络结构稳定中扮演着关键角色。
        \pause
        \item \textbf{启示:}
        这些发现为理解在线社交网络中的信息流动、影响力构建和社群组织方式提供了有价值的见解。
    \end{itemize}
\end{frame}

% --- 结束页 ---
\begin{frame}
    \centering
    \vfill
    \Huge 感谢观看!
    \vfill
    \Large Q & A
    \vfill
\end{frame}

\end{document}
